%%%%%%%%%%%%%%%%%%%%%%%%%%%%%%%%%%%%%%%%%
% Arsclassica Article
% LaTeX Template
% Version 1.1 (1/8/17)
%
% This template has been downloaded from:
% http://www.LaTeXTemplates.com
%
% Original author: 
% Lorenzo Pantieri (http://www.lorenzopantieri.net) with extensive modifications by:
% Vel (vel@latextemplates.com)
%
% License:
% CC BY-NC-SA 3.0 (http://creativecommons.org/licenses/by-nc-sa/3.0/)
%
%%%%%%%%%%%%%%%%%%%%%%%%%%%%%%%%%%%%%%%%%

%----------------------------------------------------------------------------------------
%	PACKAGES AND OTHER DOCUMENT CONFIGURATIONS
%----------------------------------------------------------------------------------------

\documentclass[
11pt, % Main document font size
a4paper, % Paper type, use 'letterpaper' for US Letter paper
oneside, % One page layout (no page indentation)
%twoside, % Two page layout (page indentation for binding and different headers)
headinclude,footinclude, % Extra spacing for the header and footer
BCOR5mm, % Binding correction
]{scrartcl}

%%%%%%%%%%%%%%%%%%%%%%%%%%%%%%%%%%%%%%%%%
% Arsclassica Article
% Structure Specification File
%
% This file has been downloaded from:
% http://www.LaTeXTemplates.com
%
% Original author:
% Lorenzo Pantieri (http://www.lorenzopantieri.net) with extensive modifications by:
% Vel (vel@latextemplates.com)
%
% License:
% CC BY-NC-SA 3.0 (http://creativecommons.org/licenses/by-nc-sa/3.0/)
%
%%%%%%%%%%%%%%%%%%%%%%%%%%%%%%%%%%%%%%%%%

%----------------------------------------------------------------------------------------
%	REQUIRED PACKAGES
%----------------------------------------------------------------------------------------

\usepackage[
nochapters, % Turn off chapters since this is an article        
beramono, % Use the Bera Mono font for monospaced text (\texttt)
eulermath,% Use the Euler font for mathematics
pdfspacing, % Makes use of pdftex’ letter spacing capabilities via the microtype package
dottedtoc % Dotted lines leading to the page numbers in the table of contents
]{classicthesis} % The layout is based on the Classic Thesis style

\usepackage{arsclassica} % Modifies the Classic Thesis package

\usepackage[T1]{fontenc} % Use 8-bit encoding that has 256 glyphs

\usepackage[utf8]{inputenc} % Required for including letters with accents

\usepackage{graphicx} % Required for including images
\graphicspath{{Figures/}} % Set the default folder for images

\usepackage{enumitem} % Required for manipulating the whitespace between and within lists

\usepackage{lipsum} % Used for inserting dummy 'Lorem ipsum' text into the template

\usepackage{subfig} % Required for creating figures with multiple parts (subfigures)

\usepackage{amsmath,amssymb,amsthm} % For including math equations, theorems, symbols, etc

\usepackage{varioref} % More descriptive referencing

%----------------------------------------------------------------------------------------
%	THEOREM STYLES
%---------------------------------------------------------------------------------------

\theoremstyle{definition} % Define theorem styles here based on the definition style (used for definitions and examples)
\newtheorem{definition}{Definition}

\theoremstyle{plain} % Define theorem styles here based on the plain style (used for theorems, lemmas, propositions)
\newtheorem{theorem}{Theorem}

\theoremstyle{remark} % Define theorem styles here based on the remark style (used for remarks and notes)

%----------------------------------------------------------------------------------------
%	HYPERLINKS
%---------------------------------------------------------------------------------------

\hypersetup{
%draft, % Uncomment to remove all links (useful for printing in black and white)
colorlinks=true, breaklinks=true, bookmarks=true,bookmarksnumbered,
urlcolor=webbrown, linkcolor=RoyalBlue, citecolor=webgreen, % Link colors
pdftitle={}, % PDF title
pdfauthor={\textcopyright}, % PDF Author
pdfsubject={}, % PDF Subject
pdfkeywords={}, % PDF Keywords
pdfcreator={pdfLaTeX}, % PDF Creator
pdfproducer={LaTeX with hyperref and ClassicThesis} % PDF producer
} % Include the structure.tex file which specified the document structure and layout

\hyphenation{Fortran hy-phen-ation} % Specify custom hyphenation points in words with dashes where you would like hyphenation to occur, or alternatively, don't put any dashes in a word to stop hyphenation altogether

%----------------------------------------------------------------------------------------
%	TITLE AND AUTHOR(S)
%----------------------------------------------------------------------------------------

\title{\normalfont\spacedallcaps{Il Dilemma del Timbro sulla Chitarra}} % The article title

\subtitle{Traduzione Italiana a cura di Davide Tedesco} % Uncomment to display a subtitle

\author{\spacedlowsmallcaps{Emilio Pujol\textsuperscript{1}}} % The article author(s) - author affiliations need to be specified in the AUTHOR AFFILIATIONS block

\date{Aprile 2020} % An optional date to appear under the author(s)

%----------------------------------------------------------------------------------------

\begin{document}

%----------------------------------------------------------------------------------------
%	HEADERS
%----------------------------------------------------------------------------------------

%\renewcommand{\sectionmark}[1]{\markright{\spacedlowsmallcaps{#1}}} % The header for all pages (oneside) or for even pages (twoside)
%\renewcommand{\subsectionmark}[1]{\markright{\thesubsection~#1}} % Uncomment when using the twoside option - this modifies the header on odd pages
%\lehead{\mbox{\llap{\small\thepage\kern1em\color{halfgray} \vline}\color{halfgray}\hspace{0.5em}\rightmark\hfil}} % The header style

%\pagestyle{scrheadings} % Enable the headers specified in this block

%----------------------------------------------------------------------------------------
%	TABLE OF CONTENTS & LISTS OF FIGURES AND TABLES
%----------------------------------------------------------------------------------------

\maketitle % Print the title/author/date block


\setcounter{tocdepth}{10} % Set the depth of the table of contents to show sections and subsections only

\newline
\tableofcontents % Print the table of contents

%\listoffigures % Print the list of figures

%\listoftables % Print the list of tables

%----------------------------------------------------------------------------------------
%	ABSTRACT
%----------------------------------------------------------------------------------------

%\section*{Abstract} % This section will not appear in the table of contents due to the star (\section*)

%\lipsum[1] % Dummy text

%----------------------------------------------------------------------------------------
%	AUTHOR AFFILIATIONS
%----------------------------------------------------------------------------------------

\let\thefootnote\relax\footnotetext{* \textit{Emili Pujol Vilarrubí (La Granadella, 7 aprile 1886 – Barcellona, 15 novembre 1980) chitarrista e compositore spagnolo.}}

%\let\thefootnote\relax\footnotetext{\textsuperscript{1} %\textit{Department of Chemistry, University of Examples, London, United Kingdom}}

%----------------------------------------------------------------------------------------

%\newpage % Start the article content on the second page, remove this if you have a longer abstract that goes onto the second page

%----------------------------------------------------------------------------------------
%	INTRODUCTION
%----------------------------------------------------------------------------------------

%\section{Introduction}

%A statement requiring citation \cite{Figueredo:2009dg}.

%\lipsum[1-3] % Dummy text

%Some mathematics in the text: $\cos\pi=-1$ and $\alpha$.
 
%----------------------------------------------------------------------------------------
%	METHODS
%----------------------------------------------------------------------------------------
\newpage

\section{Emilio Pujol e Il Dilemma del Timbro sulla\newline Chitarra }

\section{Testo originale}

%\lipsum[5] % Dummy text

%\begin{enumerate}[noitemsep] % [noitemsep] removes whitespace between the items for a compact look
%\item First item in a list
%\item Second item in a list
%\item Third item in a list
%\end{enumerate}

%------------------------------------------------

%\subsection{Paragraphs}

%\lipsum[6] % Dummy text

%\paragraph{Paragraph Description} 
\par{Sobre nada se han dado tantas definiciones como sobre las cosas definibles.} \newline
 \textsc{Becquer}
\newline
\par \textbf{\textit{La definición científica del sonido. -- El timbre en la sonoridad. -- Diferencias de timbre en cada instrumento. -- Factores queintervienen en la apreciación del sonido. Orientaciones que handeterminado su clasificación. -- Dilema del sonido en la guitarra.}}
\newline
\par{El músico que ansioso de conocer los más íntimos secretos desu arte buscara en libros científicos una definición satisfactoria delsonido, terminaría desconcertado ante la movilización de cifras ypalabras que han sido necesarias para dejar sin precisar aún, lo que sólo al espíritule es dable percibir.}\par{Esos libros nos dicen doctamente que el sonido es algo producido por lasvibraciones de un cuerpo en un medio elástico a través del cual se propaga enondas sonoras y que su timbre, intensidad y cantidad de vibraciones por segundoson extremadamente variables.}\par{Cierto que todo eso debe ser; pero algo más también, por razón de nuestrasensibilidad consciente y que no figura en esas definiciones científicas, razonadas yfrías. Un algo más variadísimo que abarca desde lo más insignificante hasta lo mástrascendental para nuestra espíritu. Un algo más que el genio del hombre puedetransformar en elemento inmaterial de un mundo maravilloso y fantástico, capaz deconfortar el alma como conforta al cuerpo un rayo de sol.}\par{La facultad de oír que nos es propia, somete el sonido a tanta diversidad deapreciaciones como diferencias de naturaleza física y moral existen entre losoyentes. Escuchar es concentrar en el oído toda nuestra sensibilidad, sensibilidadque difiere en cada individuo según su temperamento, ilustración o criterio.}\par{El oyente percibe simultáneamente con el sonido, su timbre particular (lo que  \textsc{Helmholtz} llamaba el color del sonido), envolviendo en una sola unidad suelevación, intensidad y duración.}\par{El timbre es realmente la característica de cada sonido; es lo que el color alobjeto, el perfume a la flor, la forma al cuerpo...}\par{Basta considerar la importancia que, separadamente, tiene en la orquesta cadagrupo instrumental y el timbre particular de cada instrumento. En el conjuntoarmónico, cada grupo representa un elemento determinado, subdividido en tantasindividualidades sonoras como tipos de instrumentos forman el grupo.}\par{Ninguno de estos grupos instrumentales ofrece tanta variedad de timbrescomo el de los instrumentos de cuerdas pulsadas, a causa de su diversidad deformas, tamaños, grosor y calidad de cuerdas y diferentes procedimientos en serpuestas en vibración.}\par{El timbre puede ser bueno, malo, mejor o peor, según la valorización que ledé el sentido crítico de quien lo aprecia. Como esta apreciación depende, entre milcausas, de la sensibilidad auditiva y emotiva, sugestionabilidad, educación musicale intelectual, prejuicios o fuerza de costumbre, sensatez de criterio y auncondiciones aparte, de orden general de quien lo juzga, la clasificación del timbre odel sonido, puede variar al infinito.}\par{Sin embargo, dentro del concepto relativo existe una clasificaciónpredominante que tiende a ser considerada como definitiva. Esta es la resultante deuna serie de sufragios de superior capacidad, inspirados en los principios de unaestética largamente tamizada y definida por los públicos más severos, las mejores escuelas, los mejores intérpretes y los más eminentes artífices de todos los tiempos.}\par{Este sentido es el que ha regido en la consagración de voces como las de lasfamosas cantantes GRASSINI, Jenny LYND, Adelina PATTI, la MELBA; la doscélebres GAYERRE, CARUSO, CHALIAPIN, etc.; el mismo que ha consolidad lasuperioridad de los  \textsc{STRADIVARI, AMATI} y  \textsc{GUARNERI} en los instrumentos de arco;  \textsc{BLUTHNER, BECHSTEIN, PLEYEL, ERARD} y  \textsc{STEINWAY} sobre las otras marcas depianos;  \textsc{PAGÉS, BENEDID, RECIO, ALTAMIRA} y  \textsc{TORRES} en las guitarras, y es elmismo sentido que cuida de ponderar cada artista en su sonido particular.}\par{De todos os instrumentos conocidos, ninguno habrá ofrecido seguramentemateria de tanta vacilación y discusión entre los sus adeptos, como la guitarra porla posibilidad que ofrece de ser pulsada de dos distintas maneras, con la uña y conla yema de los dedos. El timbre de la cuerda cambia sensiblemente según elprocedimiento empleado y como no cabe para unos mismos dedos la posibilidad deabarcar ambos procedimientos, el guitarrista debe adoptar uno de los dos: de ahí el dilema.} 
\newline
\par \textbf{Preferencias de timbre en la Antigüedad. Los Vihuelistas y laudistas de lossiglos XVI y XVII. Dilema de la pulsación en los guitarristas del siglo XVIII.Las teorías de SOR y de AGUADO. Posibles causas determinantes.}
\newline
\par{Desde tiempos remotísimos viene suscitando e dilema del sonidoapasionadas polémicas. Para el guitarrista, el sonido constituye una cuestióndogmática de tanta importancia como pueda ser para un moralista en sentimientode fe. Lo curioso es, que, a través del sentido estético –casi siempre intuitivo-inherente a cada partidario de un determinado timbre, podría deducirse un esbozode espiritualidad personal. Cada preferencia supone una orientación divergente,conduciendo a finalidades diametralmente opuestas.Durante la civilización griega, las preferencias oscilaban entre el sonido de lacuerda pulsada con los dedos y el que era producido por medio de un plectro.PLUTARCO refiere en su “Apothegmi Laconici” que en cierta ocasión fue castigadoun citarista por haber pulsado las cuerdas con los dedos y no con la púa durante lacelebración de una ceremonia ritual en un templo de Esparta.Sin embargo, las cuerdas pulsadas con los dedos -añade PLUTARCO-producen un sonido bastante más delicado y agradable.ATENEO, 300 años A. C., hablando de EPÍGONO dice: “fue uno de losgrandes maestros de la música; pulsaba as cuerdas con los dedos, sin plectro”.ANACREONTE y ARISTÓTELES consideraban superior también el sonido de lascuerdas pulsadas con los dedos (Ver “Precursors of the violin family”. KathleenSCHLESINGER, PÁG. 56.).Algunos aplicaban el procedimiento de pulsación al carácter de la música queinterpretaban. En una Elegía de TRIBULO, poeta de la primera centuria A. C. (LibroIII), te dice: “...y acompañándole en la citara y pulsada las cuerdas con un plectrode marfil cantó una alegre melodía con voz sonora y bien timbrada; mas después,pulsando dulcemente las cuerdas con los dedos, cantó estas palabras tristes...”.VIRGILIO en fin, dice en la Eneida (Libro VI, v. 647) “Allí danzan también encírculos, entonado un canto festivo; el bardo Traciano con sus adornos largos yflotantes, acompaña el rítmico canto con su citara de siete cuerdas pulsando ya conlos dedos, ya con un plectro de marfil.”Las preferencias entre los instrumentos de cuerda en la Edad Media seinclinaban ventajosamente en favor de los que se tocaban con los dedos o con elarco. El Arcipreste de Hita calificaba de chillones o gritadores los instrumentos decuerda de sonoridad agua y áspera.FUENLLANA, en su “Orphenica Lyra”, al hablar de los redobles1, dice: “Elherir con las uñas es imperfección... Lo que redoblan con la uña hallarán facilidaden lo que hicieren pero no perfección”. Luego añade: “Tiene gran excelencia el herirlas cuerdas con golpe sin que se entremeta uña ni otra manera de invención, puesen solo dedo, como en cosa viva consiste el verdadero espíritu, que hiriendo lacuerda se le suele dar.”Vincenzo GALILEI, refiriéndose en su “Diálogo” a la espineta, al virginal ydemás instrumentos de cuerdas metálicas, dice que ofenden grandemente el oído;no sólo por ser sus cuerdas de tal condición, sino por el objeto duro, como es lapúa, con que puestas en vibración. Prefiere francamente los instrumentos concuerdas de tripa con son el laud, la guitarra o la viola, cuya sonoridad es producidapor el contacto directo de los dedos o del arco (Véase Giuseppe BRANZOLI: “Ricerche sullo studio del liuto. Delle corde metalliche”. Roma 1889. Pág. 53.).
En el siglo XVII, época en la cual el favor por los instrumentos de mango ycuerdas pulsadas marcaba su esplendoroso apogeo, el inglés Thomas MACE,resumiendo en su valioso tratado ”The Music's Monument” la técnica que encierra elarte de sus ilustres predecesores John y Robert ROSSETER, MORLEY, CAVENDISH,COOPER, MAYNARD (“Les Luthistes”. Lionel de LA LAURENCE. Paris. 1929.) y otros,así como la de propia producción, defendía noblemente la causa del buen sonidocon estas palabras: “Adviértase que las cuerdas no deberán ser atacadas con lasuñas, como algunos hacen, pretendiendo que ésta es la mejor manera de pulsar;yo no lo creo así, por razón de que la uña, no puede obtener del laud un sonido tanpuro como el que puede producir la parte blanda y carnosa de la extremidad deldedo. Confieso que acompañado de otros instrumentos puede resultar bastantebien, ya que la principal cualidad del laud que es la suavidad. queda perdida en elconjunto; pero solo, nunca he podido lograr un resultado tan satisfactorio de lasuñas como de las yemas de los dedos”.Los tratados de guitarra aparecidos hasta fines del siglo XVIII no se detienena dar explicaciones ni expresan preferencias sobre la sonoridad; abandonan elsentido del timbre al libre albedrío del ejecutante. Solamente a partir del momentoen que la guitarra aparece montada con seis cuerdas simples, es cuando semanifiestan categóricamente las dos tendencias.AGUADO, GIULIANI, CARULLI y otros, empleaban y recomendaban la uña,mientras SOR, CARCASSI, MESSONIER y otros también, la proscribían. ¿Cómoaveriguar las causas de tales preferencias? ¿Podría ser justificación suficiente elatribuirlas simplemente al sentido estético de cada maestro? ¿Habría en ellasalguna fuerza de atavismo? SOR y AGUADO son los más explícitos al respecto.En ninguno e los comentarios biográficos de SOR, consta que hubiese pulsadonunca con uñas. Solamente en su Método - en el cual las teorías fundamentales desu técnica analizadas y razonadas se encuentran claramente explicadas – declaraque para imitar la sonoridad nasal del óboe, no solamente ataca la cuerda muyacerca del puente, sino que encorva los dedos y emplea la poca uña que tiene paraatacarla: “Es el único caso – añade – en que he creído poderme servir de ellas sininconveniente. Jamás he podido soportar un guitarrista que tocase con uñas”.Únicamente respeta la pulsación de AGUADO, en gracia a su brillanteejecución: “Era preciso que la técnica de AGUADO poseyese las excelentescualidades que posee – dice SOR- para poder perdonarle el empleo de las uñas.Desde luego, él mismo hubiera renunciado a ellas si no hubiera logrado tantaagilidad, ni se encontrase en un período de la vida en que es difícil luchar contra laacción acostumbrada de los dedos. Tan pronto oyó AGUADO algunas de mis obraslas estudió, pidiéndome consejo sobre su interpretación; pero, demasiado joven yoentonces para permitirme el derecho de corregir a un maestro de su celebridad,especialmente para mi música, concebida en una espiritualidad distinta a la quegeneralmente se inspiraban los guitarristas de entonces. Al cabo de unos cuantosaños volvimos a encontrarnos y me confesó personalmente que, si fuese arecomenzar, tocaría sin uñas”.Esta confesión no concuerda, sin embargo, con las declaraciones queAGUADO inserta en la última edición de su Método aparecido en Madrid el año1843, o sea, cuatro años después de la muerte de SOR. En él dice: “Yo siemprehabía usado de ellas (las uñas) en todos los dedos de que me sirvo para pulsar;pero, luego que escuché a mi amigo SOR, me decidí a no usarla en el dedo pulgar.Y estoy muy contento de haberlo hecho, porque la pulsación de la yema de estededo, cuando no pulsa paralelamente a la cuerda, produce sonidos enérgicos ygratos que es lo que conviene a la parte del bajo que regularmente se ejecuta enlos bordones; en los demás dedos las conservo. Como es punto del mayor interés,espero que, a lo menos por mi larga práctica se me permitirá dar mi dictamen confranqueza”. “Considero preferible tocar con uña, para sacar de las cuerdas de la guitarraun sonido que no se asemeje al de ningún otro instrumento. A mi entender laguitarra tiene un carácter particular; es dulce, armoniosa, melancólica; algunasveces llega a ser majestuosa, aunque no admite la grandiosidad del arpa ni delpiano; pero en cambio ofrece gracias muy delicadas y sus sonidos son susceptiblesde tales modificaciones y combinaciones, que la hacen parecer un instrumentomisterioso, prestándose muy bien al canto y a la expresión”. “Para producir estos efectos prefiero tocar con uñas; porque, bien usadas, elsonido que resulta es limpio, metálico y dulce; pero es necesario entender que nocon ellas solas se pulsan las cuerdas, porque no hay duda que entonces el sonidosería poco agradable. Se toca primeramente la cuerda con la yema por la parte deella que cae hacia el dedo pulgar; teniendo el dedo algo tendido (no encorvadocomo cuando se toca con la yema) y enseguida se desliza la cuerda por la uña.Estas uñas no deben ser de calidad muy dura; se han de cortar de maneraque formen una figura oval y han de sobresalir poco de la superficie de la yema,pues siendo muy largas entorpecen l agilidad, porque tarda mucho tiempo la cuerdaen salir de la uña y también hay el inconveniente de ofrecer menos seguridad en lapulsación; con ellas se ejecutan las volatas muy de prisa y con mucha claridad.”La educación musical de AGUADO no había sido la misma que la de SOR. Lade éste provenía de un ambiente severamente austero: el de la Escolanía deMonteserrat, donde además de aprender solfeo, armonía y contrapunto, estudiabael violoncelo y tomaba parte de los conjuntos vocales de música sagradapulcramente ejecutados, como ha sido siempre fama en aquel Monasterio, sin quepor todo ello abandonase su guitarra. El padre Martín que fue condiscípulo suyodurante los cinco años que SOR pasó en el Monasterio, cuenta los prodigios quehacía en su guitarra, dejando admirado a los demás compañeros y a cuantos leoían. Al abandonar el Monasterio, estudió en Barcelona el canto y lainstrumentación y estrenó con éxito en el teatro Santa Cruz su ópera “Telémaco”.Hacia 1803, siendo oficial del ejército concurrió a un concierto organizado enMálaga por el señor QUIPATRI, cónsul de Austria en aquella ciudad, y dejóasombrado a cuantos músicos e demás personas le oyeron, ejecutandobrillantemente en el contrabajo su Tema con Variaciones. (“Diccionario biográficode Efemérides de músicos españoles”. Baltasar SALDONI.)De AGUADO, nos dice Luis BALLESTEROS en su “Diccionario biográficoMatritense” que: “Desde los primeros años manifestó excelentes disposiciones parael estudio, empezando a los ocho años a estudiar Gramática latina, Filosofía yFrancés e hizo grandes adelantos en poco tiempo; dedicóse más tarde a laPaleografía, debiendo a su incansable asiduidad el título de Paleógrafo del Consejode Castilla. Por vía de distracción y recreo procuró adquirir los primeros rudimentosde la guitarra y los recibió de Fray Miguel GARCÍA, monje en el convento de SanBasilio, quien le hizo comprender los recursos y partido que podría sacar de esteinstrumento”. “El mismo SOR nos dice de su admirable colega, que sus maestros tocabancon uñas en un período en que la tendencia era de ejecutar pasajes de velocidadpara alardear de gran dominio y deslumbrar al público; no comprendían otramúsica que la que se tocaba en la guitarra y llamaban tocaban al cuarteto decuerdas, música iglesia. Afortunadamente su propio sentido personal, al orientarselibremente, le impulsó hacia una musicalidad superiormente elevada”. Sin embargo, tanto sus obras de concierto como su Método acusan unaespiritualidad más compenetrada con la brillantez de técnica que con la profundidadde emoción y elevación de concepto.Siendo SOR y AGUADO dos grandes guitarristas, la superioridad del primeroes principalmente debida al aspecto musical y artístico de su obra. A la idealidadclásica de sus Sonatas, Fantasías, Estudios y Minuetos, conviene la sonoridad sinuñas, más identificada con la música de cámara.Desde la cumbre de sus conocimientos y experiencia, en medio de unhorizonte dilatado, Sor abarca casi todos los dominios de la música. AGUADO,cautivo de su guitarra, vibra encerado en el pequeño mundo de su caja sonora,ajeno a toda expresión musical extraña.La vida de SOR fué una vida agitada e intensa; su carácter, inquieto eimpetuoso; su sensibilidad, intensamente vibrante; su temperamento, fogoso,desordenado y luchador. Viajó mucho y conoció tanto el placer embriagador de losgrandes triunfos en arte, amor y fortuna, como el dolor del fracaso, del olvido y lapobreza. Murió a los 61 años, sin fortuna e víctima de una enfermedad cruel, cuyascausas atribuyen algunos, a sus desenfrenadas pasiones.La existencia de AGUADO fue en cambio serena, afectiva y laboriosa. Sunatural inclinación al estudio , su fina sensibilidad musical y el espíritu de orden ycontinuidad en sus ideas, hicieron de él el admirable pedagogo que sabemos. Lamuerte de su madre – de quien nunca se había separado – alteró la paz de su vidaprovinciana, y se trasladó a París, donde su talento de artista y sus dotespersonales supieron conquistar la admiración de los más grandes artistas y elefecto de todos cuantos le trataron. Al regresar de este viaje a España el 12 deAbril de 1838, la diligencia en que viajaba, al llegar a Ariza (Aragón), fue asaltadapor una partida de carlistas pertenecientes al ejército de Cabrera; los cuales,después de desvalijarle, le condujeron a los montes con sus compañeros de viaje yle notificaron la sentencia de muerte, que solo podría revocarse, aprontando ciertacantidad de dinero. La misma surte amenazaba a los demás viajeros; él fue sinembargo el primero que alcanzó la libertad sin someterse a estas condiciones quese imponían para su rescate; pues su venerable ancianidad y amable trato lograronablandar el corazón de aquellos a quienes tanto había endurecido la guerra.A través de algunas cartas que hemos leído dirigidas a su colaboradorMonsieur De FOSSA, puede comprenderse toda la bondad de su carácter y lacantidad de amor e inteligencia que vertió en su obra.¿Podrían tener las opuestas preferencias de sonido en estos artistas razones deatavismo?Entre los devotos de la guitarra que han existido y existen en ciudades comoViena, Berlín, Moscú, Londres, Copenhague y otras, la mayor parte pulsan (sindarse cuenta tal vez), con la yema; mientras en otras, tocan generalmente conuñas. Familiarizados todos con su procedimiento acostumbrado, siguen inquietudesni preferencias las normas que el hábito establece.Podría admitirse pues, que el sentido de la sonoridad estuviese influenciadoen SOR por una costumbre tal vez generalizada en Cataluña; y en AGUADO segúnla costumbre de Castilla, posiblemente opuesta. Ello nos parece improbable portratarse de artistas en cuyo criterio y afán se resumen noblemente al másdesinteresado espíritu de generosidad inteligente y de sincero amor el verdaderoarte. Influencia da escuela de AGUADO, ARCAS y TÁRREGA. - El sonido “sinuñas”. - Diferencia entre el procedimiento de TÁRREGA y el de su antecesores. -Evolución artística de TÁRREGA y deducción de sus preferencias.A partir de la época en que tan alto relieve tuvieron estas prominentes figuras, elMétodo de AGUADO sirvió de guía todos los guitarristas posteriores y ha contribuidoeficazmente a que prevaleciera este sentido de sonoridad, hasta el “caso” deTÁRREGA a principios de 1900.TÁRREGA no había tocado siempre sin uñas. Los guitarristas que él conoció,incluso ASCAS, tocaban con las uñas. El tocó como ellos, sin sospechar al principiola posibilidad de una sonoridad mejor. En este período de juventud, fue cuandorealizó las campañas artísticas que le dieron fama. Pero su espíritu inquieto einvestigador tenía que chocar un día con el engorroso dilema del sonido. Ese día, nosin dudas ni titubeos, pudo decidirse, después de varias tentativas, adesembarazare de los medios que lo encumbraron, para sumergirse de pronto enlos abismos de lo incierto, en busca de mejor materia con que depurar su arte.Tuvo que privarse durante una larga temporada de tocar en público y Dios sabe sicon ello creaba a su situación precaria, dificultades angustiosas que vencer. Tuvoque trabajar a todas horas para dominar la resistencia de una nueva técnica en laque debía constituirse discípulo y maestro a la vez. Cuando fueron vencidos al finestos obstáculos, los que oímos a TÁRREGA en sus últimos tiempos no olvidaremos jamás la sensación tan pura de arte que daba su guitarra.Para obtener el sonido sin uñas que TÁRREGA lograba, no basta cortarse lasuñas a res de piel; hay que formar el sonido; es decir, desarrollar en el pulpejo delos dedos un cierto equilibrio de tacto, elasticidad y resistencia que sólo puedeconseguirse con una práctica y atención constantes.Indudablemente el sonido de SOR “sin uñas” debió ser distinto al queTÁRREGA obtenía, como debió serlo también el de AGUADO, “con uñas”, al deTÁRREGA antes de modificar su pulsación.TÁRREGA que no usaba uña ninguna atacaba generalmente la cuerda endirección perpendicular a la misma, descansando, después de la impulsión, sobre lacuerda inmediata. Este procedimiento que da un máximum de amplitud, intensidady pureza de sonido en razón de la anchura, suavidad y firmeza del cuerpo que ladesplaza no fue empleado por SOR, AGUADO ni otro de sus contemporáneos; estopuede deducirse de sus escritos y hay que suponer que, de haber sido así, lohubieran mencionado expresamente en sus respectos tratados.TÁRREGA fue influenciado en sus primeros tiempos por una época de malgusto. Sus programas se integraban como los de la mayor parte de los concertistasde otros instrumentos en su época, con obras en que la musicalidad servía tan solode pretexto a los más audaces alardes de virtuosismo. Felizmente también como elcaso de AGUADO, su temperamento, su talento y su buen gusto, lo impulsaronhacia más depurados horizontes. Su evolución significa un avance de musicalidad:MOZART, HAYDN y BEETHOVEN le apasiona y absorben primeo; CHOPIN,MENDELSSHON y SCHUMANN después; compenetrado al fin con la espiritualidad deBACH, no solamente logra el milagro de interpretar sobre las seis cuerdas simpleslas obras de este autor que mejor se adaptan a la naturaleza del instrumento (entreellas la Fuga de la Iª Sonata para violín solo) sino que su misma producciónevidencia, desde entonces, un marcado sentido de aspiración hacia la música pura.Este purismo tenía que reflejarse forzosamente en el sonido. Las cuerdaspulsadas sin uñas le ofrecieron la sonoridad soñada; un timbre puro, inmaterial yaustero. Con el trabajo obtuvo una unidad perfecta entre las notas pulsadas porcualquier dedo y en cualquier cuerda. Dominada así la materia, fue descubriendo nuevos timbres y sutilizas de ejecución que daban a sus interpretaciones mayorrelieve y persuasivo encanto.La preferencia que TÁRREGA concedió a la sonoridad sin uñas se funda enque siendo éstas materia muerta, aíslan el contacto de la sensibilidad del artistacon la cuerda. La guitarra pulsada sin las uñas viene a ser como una prolongaciónde nuestra propia sensibilidad y para un temperamento esencialmente emotivocomo era el de TÁRREGA, esta razón nos parece irrefutable.No hay atribuir la menor influencia de sentido imitativo o convencional alcambio de pulsación adoptado por TÁRREGA; fue una resolución largamentepremeditada y progresivamente definida a través de una serie de superacionessucesivas, nacidas de su ansiedad de perfección.Causas de impiden la justa apreciación del sonido. Factores queintervienen en la formación del criterio. Clasificación subjetiva del sonido.
La falta de protección oficial a que ha sido relegada casi siempre la guitarraen casi todos los países, ha hecho que en su técnica rigiera irremediablemente lamás deplorable anarquía.Para cualquier instrumento de los que en cada conservatorio se enseña,existe un método o sistema apropiado que el profesor adopta y por cual cadaalumno obtiene en resultado proporcional a sus facultades personales. Si algunadiscrepancia e establece a veces entre distintos profesores sobre particularidadestécnicas de un misma instrumento, es raro que llegue nunca a alterarse por taldisensión, el resultado general de los estudios.La enseñanza de la guitarra está ejercida la mayor parte de las veces pormaestros que estudiaron como pudieron, siguiendo libremente métodos de escuelasdefectuosas o indicaciones de maestros improvisados. Puestos a enseñar, enseñana su vez honradamente lo que saben. El discípulo que siente avidez de perfecciónse encuentra con infranqueables obstáculos; el método de técnica moderna capazde satisfacer sus aspiraciones no existe y los pocos maestros que pudieran ayudarleno son siempre accesibles.Todo condena al principiante a resignarse y buscar a ciegas el camino quemejor pueda guiarse a través de tantos escollos y dificultades.Y es natural, que un criterio formado bajo estas circunstancias seasusceptible de ignorar no solamente la importancia de una diferencia en lasonoridad, sino la diferencia misma.Nuestro sentido va educándose y definiéndose según el ambiente que nosrodea induciéndonos casi siempre a disentir de todo lo que n o concuerde con él.Solamente los escogidos logran vivir alerta sobre su propio criterio procurando quesus convicciones no les impidan comprender procurando que les son nuevos y lasopiniones justas de los demás. Las cuestiones de estética se sancionan mejor aveces a través de esa inteligencia anónima de los sentidos, que bajo el examenequilibrado de nuestra razón experimentada.Por eso, al tratar de formar una opinión sobre el sonido – inexistente sinnuestra facilidad de oír – no es posible establecer otra clasificación personal, deacuerdo con el sentido auditivo que actúa directamente sobre nuestro espíritu. Carácter físico del sonido. La clasificación en los instrumentos y en losprocedimientos. El sonido de la cuerda atacada con las uñas. El sonido delas cuerdas pulsadas con las yemas. Superioridad de la sonoridad sin lasuñas.
Intensidad, altura y timbre son las particularidades acústicas del sonido.Siempre que dos o más notas de la misma intensidad y altura produzcan en nuestrooído una sensación distinta, serán de timbre diferente.Este diferencia que puede variar al infinito, es, dentro del sentidocomparativo, susceptible de clasificación. Prueba de ello es que la categoría deciertos instrumentos de un mismo tipo está fundada en las condiciones de susonoridad. Lo que avalora principalmente los instrumentos de los artífices deCremona sobre los demás instrumentos de arco – como ocurre con las guitarras dePAGÉS, TORRES y algún otro – no es solamente la potencia sino la belleza de sussonidos.Cuerdas iguales, simultáneamente colocadas en guitarras distintas ypulsadas al aire por una misma mano en un mismo punto de la cuerda, produciránen cada instrumento una sonoridad diferente. El sonido que nos parezca mejor seráproducido seguramente por la guitarra que mejores condiciones de sonoridadtendrá; por lo tanto clasificaremos a esa guitarra de “mejor” que las demás.Ahora bien; un mismo instrumento en igualdad de condiciones no suena lomismo en manos diferentes ejecutantes. El violoncelo de CASALS, el violín deKREISLER o un piano de marca escogida, no producen la misma calidad de sonido,tratados por diferentes manos. Luego existe una categoría superior de calidadsonora en un mismo instrumento que radica en el procedimiento particular de cadaartista.El sonido de la cuerda depende: 1º de la manera de atacarla; 2º del puntodonde se la ataca y 3º de su diámetro, tensión y elasticidad.Siendo la uña un cuerpo duro de superficie, espesor y consistencia diversas,da por su impulsión a la cuerda una brillantez de timbre penetrante, espontáneo yun poco metálico aunque de amplitud escasa. El sonido sin las uñas producido porla impulsión de un cuerpo blando, sutilmente sensible y de mejor espesor yanchura, produce un timbre distinto de mayor volumen, suavidad y pureza.Lo que caracteriza las diferentes maneras de atacar la cuerda es la cantidade intensidad de los armónicos superiores que acompañan el sonido fundamental,tanto más considerables según presente el movimiento vibratorio, discontinuidadesmás numerosas y pronunciadas. Al pulsar una cuerda, el dedo la separa desuposición de reposo desde un extremo al otro de su longitud antes deabandonarla. Una discontinuidad se produce solamente en la abertura más o menosdilatada del ángulo que ofrece el sitio preciso donde ha sido impulsada por el dedo.Este ángulo es más agudo si la cuerda se ataca con la uña que cuando es atacadacon la yema del dedo. En el primer caso se obtiene un sonido más penetrante,acompañado de una gran cantidad de armónicos elevados, que tienden a metalizarsu timbre. En el segundo, las vibraciones son menos agudas, dejan de percibirsedichos armónicos y el timbre es menos brillante, más suave y más sonoro. Aunqueen ambos el sonido fundamental es más intenso que el de dichos sonidos auxiliares,a medida que se endurece el cuerpo que las ataca, se acentúan estos en detrimentodel sonido fundamental.Pruébese de atacar la cuerda con un objeto duro, uña o púa., y se observaráque el sonido que resulta es más agudo y metálico; prestando la necesaria atención de oído, se distinguirán un cantidad de notas elevadas sobre el sonido principal. Sise pulsa la misma cuerda con la yema del dedo, esas notas cesan y el sonidoresulta menos brillante, más suave y más amplio.De ahí el sentido de vaciedad del sonido con la uña (enemiga del sonidofundamental y protectora de los sonidos auxiliares), en oposición al sonido lleno ypuro de la sonoridad producida por la yema del dedo, totalmente favorable a aquélen perjuicio de los otros.El cambio de timbre de una misma cuerda bajo el mismo sistema depulsación está igualmente relacionado con la teoría de los armónicos naturales. Elsonido más puro, lo da la pulsación en la mitad de la cuerda, donde se forma elnodo de su primer armónico y es tanto más vacío y gangoso cuanto más se separade este punto hacia sus extremidades.El material de las cuerdas y su diámetro ejercen también su influencia en eltimbre. Las cuerdas muy tirantes no dan armónicos muy elevados por razón de lainflexión alternativa en pequeñas divisiones de su longitud total. Las cuerdas detripa son más ligeras y de elasticidad menos perfecta que las cuerdas recubierta demetal; de ahí que la sonoridad de aquéllas sea de menos duración que la de estasúltimas sobre todo en los armónicos naturales y en los sonidos auxiliares másagudos. (Véase “Théorie Physiologique de la Musique” H.HELMHOTZ, traducción deM. C. Guerolt, pág 105-113)Si tenemos en cuenta el sentido que ha regido para la clasificación decategorías en los instrumentos congéneres y para sonoridades producidas sobre unmismo instrumento, nos inclinaremos, evidentemente, con preferencia en favor delsonido sin las uñas, más próximo al sonido puro del arpa o del piano, y no en favordel sonido con la uña, que hace a su manera, un noble elogio de la púa. Entre unlaúd de cuerda simple y una guitarra pulsada con las uñas, poco restaría, en ciertospasajes de música movida, en favor de ésta. Y no hay duda de que, en lasuplantación del clavecín por el piano, influyó en mucho la calidad del sonidoobtenida por el martillo afelpado. Percepción del sonido según nuestro sentido psíquico. Sugestión de lasonoridad “con uñas”. Mismo aspecto de la sonoridad opuesta. El público yla técnica. Propiedades, ventajas y tendencias de cada procedimiento. Eleclecticismo en el arte. Conclusión.
Nuestras preferencias deben tener en cuenta que el sonido sirve a la músicay ésta a nuestra espiritualidad en marcha dentro de la evolución constante de lavida.Si, sustrayéndonos al sentido fríamente razonador a que nos induce laobservación del sonido bajo su aspecto físico nos abandonamos a esa divinafacultad de evocación propia del alma y sobre la cual se apoya el más elevadoconcepto de la música, cada sonoridad sugerirá en nuestra subconciencia unsentido paralelamente distinto que será el reflejo espiritual -el más persuasivo caso– de los demás aspectos del dilema.El sonido de la uña hiere el oído como si cada nota fuese una pequeña flechadiminuta y afiliada que fuese clavándose al borde de nuestra sensibilidad. Es algocónico, punzante y gangoso, heredado de la nervosidad del laúd, el monocordio y laespineta; que huele a incienso y sabe a romance antiguo; evoca retablos góticos yestilizaciones primitivas, y es como una plasmación vibrante de la ideología poéticade trovadores y plebeyos. Diríase que en las vibraciones de este timbre se encierratoda la esencia animada de un pasado lejano henchido de nobles y doradasexaltaciones del espíritu. Es la sonoridad que FALLA ha dilatado en equilibradaproporción en su “Concierto para clavecín” genial reflejo de la España austera yprofundamente cristiana de la Edad Media.El sonido de la cuerda pulsada con la yema es de una nobleza absoluta quepenetra hasta lo más recóndito de nuestra sensibilidad emotiva como penetran enel espacio el aire y la luz, sin herirlo. Las notas son inmateriales como serían las deuna arpa idealmente humanizada y confidente. Tiene en su proporción de intimidadalgo de la robustez romántica y del equilibrio griego. Evoca la gravedad del órganoy la expresividad del violoncelo. Deja ser la guitarra femenina para tomar acentosde virilidad adolescente y grave. Es en fin, la transmisión sin impurezas, de lasvibraciones más profundas de nuestra sensibilidad.El público que escucha la guitarra está en general, lejos de poder apreciarestas diferencias. Apenas si ha podido darse cuenta de las posibilidades musicales einstrumentales que la guitarra ofrece en manos de un artista. Gracias a que hayapodido y sabido recoger con su sensibilidad, la fina espiritualidad de arte queencierra. Las depuraciones de su técnica y estética, reservadas hoy a un grupolimitadísimo de conocedores, han de tardar mucho en llegar a la percepción de unagran parte del público.La uña exalta en la guitarra sus propiedades coloristas (lo que el vulgo llamaefectismos). Los armónicos pueden ser vertiginosos y acentuada la gangosidad dela cuerda; los pasajes de arpegios, escalas y ligados, rapidísimos y el rasgueado,brillante y aparatoso. Conjunto de inapreciable interés que el guitarrista debe usarcon discreción si quiere evita el peligro de incurrir en un deplorable ilusionismomusical.Como la cuerda obedece a la uña instantáneamente, permite a los dedos dela mano derecha que con un mínimum de esfuerzo obtengan el efecto deseado y enconsecuencia, la resistencia de los dedos de la mano izquierda resulte favorabledisminuida por innecesaria. Y, puesto que disminución de peso (o resistencia) essinónimo de velocidad, resultan por ello favorecidas las posiciones abiertas de los  
El dilema del sonido en la guitarra, por Emilio Pujol
dedos, los pasajes de ceja, de ligados, saltos de mano izquierda, etc., así como laprecisión claridad en las notas y la agilidad en los movimientos de ambas manos.Estas particularidades permiten al guitarrista realizar con menos dificultad ymayor brillantez las proezas de agilidad que una parte del público gusta admirar,dándole a la vez la sensación de mayor dominio. Y esta admiración que el públicocomunica al artista en virtud de la corriente hipnótica que se establece entre laatención pasiva del oyente y la atención activa del ejecutante, es la mejorafirmación en éste, de la confianza y fe en sí mismo.Las particularidades que ofrece la pulsación sin uñas son otras. El volumen,uniformidad y fusión de notas a través de toda la extensión de sus cuerdas recogeny encauzan toda la variedad de sus matices en un sentido de sobria musicalidad.Loas acordes dan máximum de unidad, intensidad y volumen; el trémolo deja deser metálico y brillante para transformarse en sonoridad etérica y velada; elpizzicato obtiene toda su agudeza y carácter en todas las cuerdas y los arpegios yescalas consiguen todo el volumen, fusión y regularidad de proporción entre susnotas. Esta pulsación se presta poco a ciertos efectos espectaculares; al contrario,el artista, aunque encuentra en ella todos los elementos necesarios de expresión,debe poner oportunamente en juego sus recursos si quiere evitar la unidadpersistente degenere en monotonía.Siendo la yema un cuerpo blando más ancho que la uña, al desplazar deimpulsión exige a la mano izquierda mayor esfuerzo y este esfuerzo de impulsiónexige a la mano izquierda mayor presión y resistencia para las notas pisadas. Daahí, que cualquier pasaje de cejas, ligados, posiciones abiertas o forzadas y ciertospasajes de virtuosismo resulten más difíciles de vencer.En cada procedimiento cabe pues, una espiritualidad distinta; la unaespectacular tendiendo a la exteriorización personal y la otra intimista y sincera, decompenetración profunda con el arte.No hay que olvidar que por encima de todas las propiedades que ofrece laguitarra, la más importante y en la que ningún otro instrumento probablemente leaventaja, es la de su poder de adaptación a la espiritualidad del arte que traduce.El eclecticismo en arte puede milagrosamente trocar los defectos envirtudes; del mismo modo que una sonoridad austera puede ser adecuada para unamusicalidad severa, una sonoridad brillante puede dar más autenticidad a ciertamúsica de carácter o estilización particular.Sería lamentable cerrar en la exclusividad en criterio que acabase connuestro viejo dilema. Lo que cuenta en materia de arte es el espíritu. Felicitémonospues de que la guitarra ofrezca esa dualidad de aspectos, en la cual, cada artistapueda, según sus sentimientos realizar su obra con sinceridad recogiendo a travésde ella, la justa admiración que corresponda a sus méritos.}

%\paragraph{Different Paragraph Description} \lipsum[8] % Dummy text

%------------------------------------------------

%\subsection{Math}

%\lipsum[4] % Dummy text

%\begin{equation}
%\cos^3 \theta =\frac{1}{4}\cos\theta+\frac{3}{4}\cos 3\theta
%\label{eq:refname2}
%\end{equation}

%\lipsum[5] % Dummy text

%\begin{definition}[Gauss] 
%To a mathematician it is obvious that
%$\int_{-\infty}^{+\infty}
%e^{-x^2}\,dx=\sqrt{\pi}$. 
%\end{definition} 

%\begin{theorem}[Pythagoras]
%The square of the hypotenuse (the side opposite the right angle) %is equal to the sum of the squares of the other two sides.
%\end{theorem}

%\begin{proof} 
%We have that $\log(1)^2 = 2\log(1)$.
%But we also have that $\log(-1)^2=\log(1)=0$.
%Then $2\log(-1)=0$, from which the proof.
%\end{proof}

%----------------------------------------------------------------------------------------
%	TRADUZIONE ARTICOLO
%----------------------------------------------------------------------------------------
\newpage
\section{Traduzione}

%Reference to Figure~\vref{fig:gallery}. % The \vref command specifies the location of the reference

%\begin{figure}[tb]
%\centering 
%\includegraphics[width=0.5\columnwidth]{GalleriaStampe} 
%\caption[An example of a floating figure]{An example of a floating figure (a reproduction from the \emph{Gallery of prints}, M.~Escher,\index{Escher, M.~C.} from \url{http://www.mcescher.com/}).} % The text in the square bracket is the caption for the list of figures while the text in the curly brackets is the figure caption
%\label{fig:gallery} 
%\end{figure}

\par\textit{"Nessun soggetto è descritto da una tale ricchezza di definizioni come averne indefinite".}  \newline \textsc{Becquer}
\newline
\par \textbf{\textit{La definizione scientifica del suono. Il timbro nella sonorità. Differenze di timbro in ogni strumento. Fattori che definiscono l'apprezzamento di un suono. Tendenze che ne determinano la classificazione. Dilemma del Timbro sulla Chitarra.}}
\newline
\par{Il musicista studioso, ansioso di scoprire il piú interno segreto della sua arte, cercherebbe invano una definizione soddisfacente di suono in libri di materia scientifica. La sua ricerca terminerebbe in modo deludente prima di esser sconcertato da un insieme di parole e figure pensate per spiegare un qualcosa che solamente lo spirito puó penetrare.} \par{Questi libri ci dicono, con linguaggio alti-sonante, e metodologie scolastiche, che il suono è un qualcosa prodotto dalle vibrazioni di un corpo in un mezzo elastico attravero il quale si propagano le onde sonore, e che il timbro o qualità, intesità, volume e numero di vibrazioni(N.d.T. al secondo) sono estremamente variabili. Ovviamente, tutto ciò è essere corretto, per quanto lo puó essere, ma è altrettanto incompleto. C'è qualcos'altro in piú di ció, qualcosa di cui la nostra sensibilità cosciente è al corrente, che non appare nelle lunghe e aride definizioni puramente scientifiche. Qualcosa la cui natura varia eccessivamente, abbracciando i piú insignificanti ma allo stesso tempo trascendentali aspetti della nostra mente.} \par{Qualcosa che la mente di un uomo puó trasformare in un mondo immateriale fatto di meraviglia e fantasia, un mondo capace di animare e confortare lo spirito come l'alba puó confortare il corpo umano. La facoltà di sentire cosa è naturale per noi, è soggetta alla diversità della natura fisica e morale dell'ascoltatore. Ascoltare significa concentrarare tutta la sensibilità sul potere dell'ascolto, una sensibilità che è differente per ogni individuo secondo il proprio temperamento personale, l'esperienza e la capacità di discernimento.} \par{L'ascoltatore percepisce simultaneamente il suono, in particolare il timbro (il colore del suono, come Helmoltz lo ha definito) e lo sperimenta, con la sua altezza(N.d.T. il suo pitch), intesità e durata in un singolo impatto. Il timbro è, infatti, l'elemento caratteristico di ogni suono e come il colore di un oggetto, il profumo di un fiore, forma il corpo, come il timbro forma il suono. Quando consideriamo, separatamente, l'importanza di ogni gruppo strumentale come un tutt'uno in un'orchestra, insieme con le particolari qualità di ogni tipologia di strumento, troviamo che ognuno di questi gruppi rappresenti un elemento definito, costruito dalle tonalità individuale di ogni singolo strumento. Ora, nessun gruppo orchestrale puó produrre una tale varietà di timbri come quella degli strumenti pizzicati, tenendo conto della diversità di forme e dimensioni, lo spessore, tipe e qualità delle corde, e differenti metodi utilizza per farle vibrare. Il timbro puó esser considerato buono o cattivo, migliore o peggiore in proporzione al maggiore o minore valore delineato da ogni singolo critico. Il giudizio della critica puó essere affetto da innumerabili cause, come la sensibilità uditiva e sensioriale, l'impressionabilità, l'educazione musicale ed intellettuale, la forza dell'abitudione, il pregiudizio, la tradizione e l'ambiente; dunqe la classificazione del suono puó variare enormemente. Noi abbiamo, oltretutto, in dei certi limiti, una prevalente classificazione che puó esser considerata una guida affidabile: il risultato dei principi estetici è gradualmente evoluto, venendo setacciato accuratamente, ed infine stabilito dagli studiosi piú esigenti e dai eminenti interpreti e artigiani. L'applicazione cosciente di questi principi ha portato all'unversale riconoscimento delle voci di famosi cantanti come Grassini, Jenny Lynd, Adelina Patti, Melba, Gayarre, Caruso, Chaliapin, etc. Lo stesso spirito ha stabilito fermamente la superiorità di Stradivari, Amati e Guarnieri nel campo degli strumenti ad arco; similmente, i pianoforti realizzati da Bluthner, Bechstein, Pleyel, Erard, e Steinway sono divenuto quelli di prim'ordine, e per le chitarre di Pagés, Benedit, Recio, Altamira e Torres. Simili principi guidano un artista nella qualità del tono che egli deve cercare di realizzare. Degli strumenti ad oggi conosciuti, sicuramente nessuno di essi ha mai offerto maggior materiale di controversia della chitarra. Questo è principalmente a causa della possibilità di pizzicare le corde in due modi distini: con le unghie o con la parte di carne(N.d.T. il polpastrello) alla punta del dito. Il suono differisce a seconda della tecnica utilizzata e non è possibile per lo stesse dita usare due tecniche, il musicista deve prendere una decisione: ecco il dilemma.}
\newline
\par \textbf{\textit{Preferenze di timbro tra gli Antichi. I liutisti del 17esimo Secolo. Il dilemma del modo di suonare tra i chitarristi del 18esimo Secolo. Le teorie di Sor ed Aguado. Probabili cause determinanti.}}
\newline
\par{Dal piú antico dei tempi in cui si presentò il dilemma del suono esso divenne causa di appassionate contrversie. Per il chitarrista la questione del suono è importante come un articolo di fede per un moralista. È interessante notare che il sentimento estetico -intuitivo, in quasi ogni caso - inerente ogni parte di un particolare timbro, rivela, in grande estensione la sua personalità. A qualsiasi scelta fatta, é implicato un punto di vista mentale divergente che porta talvolta a conclusioni diametricalmente opposte. Tra gli antichi Greci, si preferirono due stili distinti di percuotere le corde, alcuni musicisti usavano le dita, ed altri un adoperavano un plettro. Negli "Apothegmi Laconici", Plutarco riferisce che una volta, a Sparta, un chitarrista fu punito per aver pizzicato le corde con le dita, e non con un plettro, durante una celebrazione di una cerimonia rituale in un tempio. E ancora, egli aggiunge, che le corde pizzicate con le dita produce un suono piú dolce e piacevole del plettro. Ateneo, nel 300 a.C. riferendozi a Epigono, dice, "Lui era uno dei piú grandi maestri di musica; pizzicava le corde con le dita e non con il plettro". Aristogone ed Anacreonte consideravano il suono delle corde pizzicate con le dita migliore di quello prodotto utilizzando un plettro.} \par{Alcuni musicisti usavano tutti e due gli stili, a seconda del carattere di musica che essi dovevano interpretare. Tibullio in una delle sue elegie, (Terzo Libro, Elegie 4, v. 39), descrive in un passaggio, liberamente tradotto, dice: "E accompagnando la sua stessa voce con una chitara e percuotendo le corde con un plettro d'avorio, e cantando una melodia allegra a voce alta e squillante; ma dopo cioe, pizzicando le corde delicatamente con le sue dita, cantò queste parole..." Infine, Virgilio, nell'Eneide, dice "Lí essi danzavano in cerchio mentre cantavano una gioiosa canzone; il bardo di Tracia, nei suoi lunghi ed comodi abiti, accompagnava il ritmo su di una cithare a sette corde, ora pizzicando le corde con le dita, ora con il plettro." Fortunatamente, nel Medioevo, gli strumenti a corda, suonati sia con le dita che con un arco, divennero quelli piú utilizzati. Arcipreste de Hita definí come "fastidiosi e striduli" gli strumenti a corda con uno acuto ed aspro tono.}

%------------------------------------------------

%\subsection{Subsection}

%\lipsum[11] % Dummy text

%\subsubsection{Subsubsection}

%\lipsum[12] % Dummy text

%\begin{description}
%\item[Word] Definition
%\item[Concept] Explanation
%\item[Idea] Text
%\end{description}

%\lipsum[12] % Dummy text

%\begin{itemize}[noitemsep] % [noitemsep] removes whitespace %between the items for a compact look
%\item First item in a list
%\item Second item in a list
%\item Third item in a list
%\end{itemize}

%\subsubsection{Table}

%\lipsum[13] % Dummy text

%\begin{table}[hbt]
%\caption{Table of Grades}
%\centering
%\begin{tabular}{llr}
%\toprule
%\multicolumn{2}{c}{Name} \\
%\cmidrule(r){1-2}
%First name & Last Name & Grade \\
%\midrule
%John & Doe & $7.5$ \\
%Richard & Miles & $2$ \\
%\bottomrule
%\end{tabular}
%\label{tab:label}
%\end{table}

%Reference to Table~\vref{tab:label}. % The \vref command specifies the location of the reference

%------------------------------------------------

%\subsection{Figure Composed of Subfigures}

%Reference the figure composed of multiple subfigures as Figure~\vref{fig:esempio}. Reference one of the subfigures as Figure~\vref{fig:ipsum}. % The \vref command specifies the location of the reference

%\lipsum[15-18] % Dummy text

%\begin{figure}[tb]
%\centering
%\subfloat[A city %market.]{\includegraphics[width=.45\columnwidth]{Lorem}
%} \quad
%\subfloat[Forest landscape.]{\includegraphics[width=.45\columnwidth]{Ipsum}\label{fig:ipsum}} \\
%\subfloat[Mountain landscape.]{\includegraphics[width=.45\columnwidth]{Dolor}} \quad
%\subfloat[A tile decoration.]{\includegraphics[width=.45\columnwidth]{Sit}}
%\caption[A number of pictures.]{A number of pictures with no common theme.} % The text in the square bracket is the caption for the list of figures while the text in the curly brackets is the figure caption
%\label{fig:esempio}
%\end{figure}

%----------------------------------------------------------------------------------------
%	BIBLIOGRAPHY
%----------------------------------------------------------------------------------------

\renewcommand{\refname}{\spacedlowsmallcaps{References}} % For modifying the bibliography heading

\bibliographystyle{unsrt}

\bibliography{sample.bib} % The file containing the bibliography

%----------------------------------------------------------------------------------------

\end{document}